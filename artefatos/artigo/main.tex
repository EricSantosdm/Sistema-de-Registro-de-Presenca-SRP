\documentclass[conference]{IEEEtran}
\IEEEoverridecommandlockouts
% The preceding line is only needed to identify funding in the first footnote. If that is unneeded, please comment it out.
\usepackage[utf8]{inputenc}  % Coloque o inputenc antes de outros pacotes que lidam com texto
\usepackage[brazil]{babel} 
\usepackage{cite}
\usepackage{amsmath,amssymb,amsfonts}
\usepackage{algorithmic}
\usepackage{graphicx}
\usepackage{textcomp}
\usepackage{xcolor}
\usepackage{url}

% \usepackage[alf, abnt-emphasize=bf, recuo=0cm, abnt-etal-cite=3, abnt-etal-list=0, abnt-etal-text=it]{abntex2cite}

\begin{document}

\title{Sistema de Registro de Presença}

\author{\IEEEauthorblockN{BRUNO WELLINGTON DA SILVA LIMA}
\IEEEauthorblockA{\textit{UFERSA}\\
Pau dos Ferros, Brasil \\
bruno.lima50028@alunos.ufersa.edu.br}
\and
\IEEEauthorblockN{DANIEL LINS SILVA ANDRADE}
\IEEEauthorblockA{\textit{UFERSA}\\
Pau dos Ferros, Brasil \\
daniel.loading404@gmail.com}
\and
\IEEEauthorblockN{DIMONA LAQUIS ALVES ANDRADE}
\IEEEauthorblockA{\textit{UFERSA}\\
Pau dos Ferros, Brasil \\
dimona.andrade@alunos.ufersa.edu.br}
\and
\IEEEauthorblockN{EMANUELA BEZERRA DE LIMA}
\IEEEauthorblockA{\textit{UFERSA}\\
Pau dos Ferros, Brasil \\
emanuela.lima@alunos.ufersa.edu.br}
\and
\IEEEauthorblockN{ERIC DOS SANTOS BEZERRA}
\IEEEauthorblockA{\textit{UFERSA}\\
Pau dos Ferros, Brasil \\
eric094@outlook.com}
\and
\IEEEauthorblockN{FRANCISCO FLÁVIO NOGUEIRA DA SILVA}
\IEEEauthorblockA{\textit{UFERSA}\\
Pau dos Ferros, Brasil \\
francisco.silva29241@alunos.ufersa.edu.br}
\and
\IEEEauthorblockN{NATÁLIA VITÓRIA MOURA DA SILVA}
\IEEEauthorblockA{\textit{UFERSA}\\
Pau dos Ferros, Brasil \\
natalia.silva@alunos.ufersa.edu.br}
}

\maketitle

\begin{abstract}
O resumo deve responder os seguintes pontos:
\begin{itemize}
    \item Qual o contexto?
    \item Qual o problema?
    \item Qual a relevância?
    \item Qual a sua contribuição?
    \item Quais as conclusões (os achados)?
\end{itemize}
\end{abstract}

\begin{IEEEkeywords}
key1, key, ..., keyn
\end{IEEEkeywords}

\section{Introdução}
\label{sec:introducao}

Texto da introdução aqui.

\section{Fundamentação teórica}
\label{sec:fundamentacao_teorica}

Texto da fundamentação teórica aqui.

\section{Abordagem}
\label{sec:abordagem}

Este trabalho aborda o desenvolvimento de um sistema automatizado de registro de presença via \textit{QR code}, com o objetivo de otimizar o controle de presença em ambientes acadêmicos. A pesquisa inicial foi bibliográfica e exploratória, focando no contexto do registro de presença, identificando tendências, desafios e soluções relevantes.

Durante o desenvolvimento, foram definidos os requisitos funcionais e técnicos, escolhendo as tecnologias apropriadas. O \textit{backend} foi construído em \textit{Python} com o \textit{framework} \textit{Django}, enquanto o \textit{frontend} utilizou \textit{HTML}, \textit{CSS}, \textit{JavaScript} e o \textit{framework} \textit{Tailwind}, assegurando uma interface intuitiva e responsiva. O gerenciamento de versão do código foi feito com Git e GitHub.

Para a infraestrutura e o deploy, foi empregada a plataforma \textit{AWS (Amazon Web Services)}, que oferece escalabilidade e segurança. A qualidade do código foi garantida por testes automatizados com \textit{Pytest} e ferramentas de análise como \textit{SonarCloud} e \textit{SonarQube}, que monitoraram vulnerabilidades e propuseram melhorias no código.

\section{Trabalhos Relacionados}
\label{sec:trabalhos_relacionados}

Aranega e Pera \cite{aranega2023} desenvolveram o aplicativo Univap Presença Digital, que utiliza QR codes exclusivos para otimizar a autenticação e a contagem de presença de alunos em escolas. Integrado a uma catraca inteligente e um backend em Node.js, o sistema melhorou significativamente a eficiência e a segurança no controle de acesso, eliminando atrasos e permitindo uma análise detalhada das frequências escolares. A implementação trouxe avanços notáveis na gestão do tempo e na experiência educacional, além de promover práticas mais sustentáveis no ambiente escolar.

Romera e Honorato \cite{romera2023} desenvolveram um aplicativo de registro de presença digital utilizando reconhecimento facial e QR codes. Implementado com tecnologias como Python, React e PostgreSQL, o sistema visa automatizar o controle de presença em ambientes corporativos e acadêmicos, melhorando a eficiência e segurança no processo. Apesar de resultados promissores, o sistema apresentou desafios relacionados à iluminação e à qualidade das imagens no reconhecimento facial, além de questões de segurança no uso dos QR codes.

Paula e Oliveira \cite{paula2023} desenvolveram uma plataforma web para a gestão de eventos acadêmicos, utilizando Laravel, Vue.js e PHP. A plataforma permite o cadastro de eventos e participantes, autenticação de presença via QR code, e a geração de certificados. A solução foi projetada para facilitar a organização dos eventos, reduzir erros e melhorar a eficiência do processo, proporcionando uma experiência mais prática tanto para organizadores quanto para participantes.


\section{Consideracoes finais e trabalhos futuros}
\label{sec:consideracoes_finais_trabalhos_futuros}

% Texto das considerações finais e trabalhos futuros aqui.

O objetivo central deste trabalho foi desenvolver e implementar um sistema automatizado para registro de presença via QR code, visando otimizar o processo de controle de presença em ambientes acadêmicos. Após o levantamento dos requisitos, a seleção criteriosa das tecnologias e a execução do desenvolvimento, o sistema foi concluído e encontra-se plenamente operacional. A seguir, são apresentados os resultados obtidos com a sua aplicação.


Apenas um exemplo de referência: \cite{aurelio}.

\bibliographystyle{acm}
\bibliography{references}

\end{document}
