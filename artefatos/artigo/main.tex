\documentclass[conference]{IEEEtran}
\IEEEoverridecommandlockouts
% The preceding line is only needed to identify funding in the first footnote. If that is unneeded, please comment it out.
\usepackage[utf8]{inputenc}  % Coloque o inputenc antes de outros pacotes que lidam com texto
\usepackage[brazil]{babel} 
\usepackage{cite}
\usepackage{amsmath,amssymb,amsfonts}
\usepackage{algorithmic}
\usepackage{graphicx}
\usepackage{textcomp}
\usepackage{xcolor}
\usepackage{url}

% \usepackage[alf, abnt-emphasize=bf, recuo=0cm, abnt-etal-cite=3, abnt-etal-list=0, abnt-etal-text=it]{abntex2cite}

\begin{document}

\title{Título do Projeto}

\author{\IEEEauthorblockN{Fulano}
\IEEEauthorblockA{\textit{Organização}\\
Pau dos Ferros, Brasil \\
email}
\and
\IEEEauthorblockN{Sicrano}
\IEEEauthorblockA{\textit{UFERSA}\\
Pau dos Ferros, Brasil \\
email}
\and
\IEEEauthorblockN{Beltrano}
\IEEEauthorblockA{\textit{UFERSA}\\
Pau dos Ferros, Brasil \\
email}
}

\maketitle

\begin{abstract}
O resumo deve responder os seguintes pontos:
\begin{itemize}
    \item Qual o contexto?
    \item Qual o problema?
    \item Qual a relevância?
    \item Qual a sua contribuição?
    \item Quais as conclusões (os achados)?
\end{itemize}
\end{abstract}

\begin{IEEEkeywords}
key1, key, ..., keyn
\end{IEEEkeywords}

\section{Estrutura}
\label{sec:intro}

O artigo deve conter as seções a seguir:

\begin{itemize}
    \item Introdução
    \item Fundamentação teórica f
    \item Sua abordagem
    \item Considerações finais e Trabalhos Futuros
\end{itemize}

Apenas um exemplo de referência: \cite{aurelio}.

\bibliographystyle{acm}
\bibliography{references}

\end{document}
