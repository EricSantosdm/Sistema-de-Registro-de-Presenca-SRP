\section{Resultados e discussões}
\label{sec:resultados_discussoes}

O objetivo deste trabalho foi desenvolver um sistema automatizado para registro de presença via \textit{QR code}, visando otimizar o controle em ambientes de eventos. A implementação permitiu um registro de presença ágil e preciso, eliminando processos manuais e possibilitando a captura em tempo real, com o uso do \textit{framework} \textit{Django} que foi a ferramenta central do desenvolvimento. \textit{Feedbacks} positivos destacaram a interface intuitiva, desenvolvida com \textit{HTML}, \textit{CSS} e \textit{Tailwind}, que facilitou o acesso e uso do sistema pelos participantes.

Desafios como a integração tecnológica e segurança da informação foram abordados. Testes automatizados com \textit{Pytest} e ferramentas como \textit{SonarCloud} e \textit{SonarQube} garantiram a qualidade do código e proteção contra vulnerabilidades.

A automação do registro de presença pode transformar a gestão de eventos, permitindo a geração de relatórios detalhados sobre frequência, facilitando a identificação de padrões e o acompanhamento dos participantes.
