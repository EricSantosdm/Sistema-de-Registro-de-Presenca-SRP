\section{Fundamentação teórica}
\label{sec:fundamentacao_teorica}

\subsection{Introdução aos Sistemas de Registro de Presença}
Sistemas de registro de presença são amplamente utilizados em diversos contextos, como eventos corporativos, conferências, workshops, escolas, e até em reuniões online \cite{ferreira2023}. Eles permitem o controle da participação, oferecendo dados valiosos sobre quem compareceu e quem estava programado para estar presente. Além disso, esses sistemas podem ser usados para automatizar processos de registro, como a emissão de certificados de participação, geração de relatórios de presença, e até mesmo a coleta de feedbacks dos participantes.

\subsubsection{Evolução e Aplicação}
Sistemas de registro de presença não são novos, mas evoluíram significativamente com o tempo. No passado, eram manuais, baseados em listas de papel onde os participantes assinavam sua presença. A digitalização trouxe novas formas de registrar presença \cite{qureshi2024}, como:
\begin{itemize}
    \item Dispositivos biométricos.
    \item RFID (radiofrequência).
    \item Códigos QR.
    \item Aplicativos e plataformas web.
\end{itemize}

Sua aplicação pode ser diversa, como:
\begin{itemize}
    \item \textbf{Eventos corporativos:} controle de presença em treinamentos, workshops, e reuniões.
    \item \textbf{Escolas e universidades:} controle de presença em aulas, seminários, e eventos acadêmicos.
    \item \textbf{Conferências e feiras:} controle de presença em palestras e estandes.
    \item \textbf{Reuniões online:} controle de presença em videoconferências, webinars, e reuniões virtuais.
\end{itemize}

\subsubsection{Funcionalidades e Tecnologias}
Sistemas de presença modernos podem oferecer uma série de funcionalidades, dependendo da complexidade e do público-alvo:

\begin{itemize}
    \item \textbf{Registro pré-evento:} sistemas que permitem inscrições, garantindo que o organizador tenha uma lista prévia dos participantes.
    \item \textbf{Check-in/out:} os participantes confirmam sua presença via QR code, aplicativos, dispositivos RFID, reconhecimento facial ou outros métodos.
    \item \textbf{Relatórios e análise de dados:} esses sistemas permitem gerar relatórios detalhados com dados como horário de chegada, duração da participação, e número de ausentes, oferecendo insights para os organizadores.
    \item \textbf{Integração com outras plataformas:} alguns sistemas se integram com sistemas de gestão de eventos, CRM, e até plataformas de pagamento.
\end{itemize}

\subsection{Fundamentos de Sistema Web}

\subsubsection{Sistemas web}
A Web, criada em 1989 por Tim Berners-Lee, facilita a troca de informações na Internet. Os sistemas web são aplicações acessíveis via navegadores, operando em ambientes distribuídos e geralmente compostos por HTML e linguagens como PHP ou Java. 
Ferreira \cite{ferreira2023} destaca que, ao desenvolver sistemas web, é crucial considerar um conjunto de atributos essenciais:

\begin{itemize}
    \item \textbf{Usabilidade:} experiência acessível e rápida.
    \item \textbf{Funcionalidade:} busca eficiente e navegação intuitiva.
    \item \textbf{Confiabilidade:} funcionamento correto e validação de usuários.
    \item \textbf{Manutenibilidade:} facilidade de manutenção e atualização.
\end{itemize}

\subsubsection{Arquitetura Cliente/Servidor}
Desenvolvida na década de 1980, a arquitetura cliente/servidor divide funções entre clientes (que solicitam serviços) e servidores (que atendem a essas solicitações). Exemplos incluem bancos de dados e servidores de aplicação, que processam e retornam dados aos usuários. \cite{ferreira2023}.

\subsection{QR Codes}
O QR Code (Quick Response) representa uma evolução significativa em relação ao código de barras, revolucionando o acesso à informação \cite{okada2011}. Ele armazena dados bidimensionalmente, permitindo leitura a partir de qualquer direção. Após a decodificação, pode exibir texto ou links que redirecionam para conteúdo online.

A criação de QR codes é realizada de forma simples e ágil. Sua leitura pode ser realizada por meio de um aplicativo específico no smartphone ou tablet; basta apontar a câmera e o conteúdo, como links ou mapas, é acessado rapidamente \cite{bernardo2011}.

Uma das principais vantagens do QR Code é sua praticidade, por eliminar a necessidade de digitar endereços da web, simplificando o acesso a informações \cite{goncalves2014}.

\subsection{Automatização de Processos}
A automatização de processos visa otimizar atividades em uma organização, reduzindo esforço, tempo e custos, além de substituir tarefas manuais por software \cite{roig2017}. Segundo Roig, ``a automação alia tecnologia da informação e gerenciamento de negócios para otimizar resultados e alcançar objetivos globais'', resultando em produção mais ágil e maior satisfação do cliente.

Além disso, a automação oferece benefícios como acesso rápido à informação, gestão eficaz, e monitoramento em tempo real. Ela ajuda a aumentar a produtividade ao identificar problemas rapidamente, melhora a comunicação, e eleva a competitividade da organização. Também contribui para a qualidade de vida dos colaboradores, reduzindo o retrabalho e facilitando a rastreabilidade dos processos.

\subsection{Introdução ao Desenvolvimento com \textit{Python}}
\textit{Python} é uma das linguagens dinâmicas mais populares e poderosas, apoiada por uma grande comunidade global \cite{borges2010}. Conhecida por sua sintaxe clara e legibilidade, é fácil de aprender, mesmo apresentando conceitos técnicos complexos. Desenvolvida em 1990 por Guido van Rossum, \textit{Python} foi inicialmente projetada para físicos e engenheiros, e seu nome foi inspirado no grupo de comédia Monty Python.

Como uma linguagem orientada a objetos, \textit{Python} é versátil, sendo utilizada em diversos domínios, desde scripts simples até aplicações complexas \cite{lutz2007}. Sua sintaxe simples, tipagem dinâmica e a ausência de etapas de compilação otimizam o desenvolvimento, permitindo que programadores criem soluções rapidamente. Essas características aumentam a produtividade e fazem de \textit{Python} uma escolha popular para desenvolvedores em busca de eficiência e profundidade técnica.

\subsection{\textit{Frameworks} de Desenvolvimento \textit{Web}: \textit{Django}}
\textit{Django} é um \textit{framework} de desenvolvimento \textit{web} em \textit{Python} que proporciona uma estrutura eficiente para criar aplicações com código limpo e sustentável, evitando a reinvenção de funcionalidades comuns \cite{meireles2010}. Baseado na arquitetura \textit{Model-View-Controller (MVC)}, é um software livre que acelera o desenvolvimento de sites dinâmicos e serviços \textit{web}.

O \textit{framework} inclui um sistema de templates que separa a lógica de programação do design, permitindo a construção ágil de sites e aliviando os programadores de tarefas repetitivas. Isso permite que eles se concentrem em aspectos mais inovadores do desenvolvimento. \textit{Django} oferece abstrações de alto nível para padrões comuns, tornando o processo mais produtivo, embora exija conhecimentos básicos de desenvolvimento web, programação orientada a objetos, bancos de dados e \textit{Python}.

Lançado como projeto de código aberto em 2005, \textit{Django} foi inicialmente criado para gerenciar um site jornalístico e seu nome é uma homenagem ao músico de jazz Django Reinhardt, simbolizando agilidade e inovação \cite{meireles2010}. É amplamente adotado devido à sua capacidade de acelerar o desenvolvimento e à clareza que oferece na resolução de problemas comuns.

\subsection{Engenharia de Requisitos}
A engenharia de requisitos é o processo de entender e definir os serviços e restrições de um sistema, sendo essencial para o sucesso do desenvolvimento de software \cite{sommerville2011}. Uma documentação adequada nesta fase é crucial, pois erros podem impactar negativamente o projeto e a implementação. Os requisitos descrevem o que o sistema deve fazer, incluindo serviços e limitações.
Os requisitos se dividem em dois tipos principais \cite{valente2020}:

\begin{itemize}
    \item \textbf{Funcionais:} especificam o que o sistema deve fazer, ou seja, as funcionalidades ou serviços a serem implementados.
    \item \textbf{Não funcionais:} descrevem como o sistema deve operar, incluindo restrições e qualidade do serviço, como desempenho, segurança e usabilidade.
\end{itemize}

O objetivo da engenharia de requisitos é criar um documento que sintetize as necessidades dos stakeholders, servindo como uma declaração oficial do que deve ser implementado, incluindo requisitos de usuários e especificações técnicas \cite{sommerville2011}.

\subsection{Modelagem e Diagramas}

\subsubsection{Introdução à modelagem de sistemas}
A modelagem de dados, segundo Cougo  \cite{cougo1997}, é o processo de criar diagramas que simplificam a visão de sistemas complexos, focando no fluxo de dados e definindo atributos de dados e relacionamentos entre entidades. Isso facilita a comunicação e a tomada de decisões durante o ciclo de desenvolvimento.

\subsubsection{Diagramas UML: Casos de uso, classes, modelo lógico de dados}
A \textit{UML (Unified Modeling Language)} é uma linguagem que descreve projetos de software, oferecendo uma visão clara e estruturada, especialmente em ambientes orientados a objetos \cite{fowler2005}. Três diagramas UML são particularmente importantes:

\begin{itemize}
    \item Descreve a estrutura estática do sistema, detalhando tipos de objetos, suas propriedades, operações e relacionamentos. Isso ajuda na definição e implementação do design do software.
    \item Mostra as funcionalidades do sistema e as interações dos usuários com essas funcionalidades, apresentando uma visão clara do que o sistema deve realizar \cite{sommerville2011}.
    \item Explora a estrutura e os relacionamentos dos dados, servindo como base para a criação do modelo de dados físico em projetos \cite{cougo1997}.
\end{itemize}

Esses diagramas são fundamentais para garantir uma compreensão compartilhada do sistema, apoiando um desenvolvimento eficiente alinhado às necessidades dos usuários.
