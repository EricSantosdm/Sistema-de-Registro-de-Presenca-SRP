\section{Fundamentação teórica}
\label{sec:fundamentacao_teorica}

\subsection{Introdução aos Sistemas de Registro de Presença}
Sistemas de registro de presença são amplamente utilizados em diversos contextos, como eventos corporativos, conferências, workshops, escolas, e até em reuniões online \cite{ferreira2023}. Eles permitem o controle da participação, oferecendo dados valiosos sobre quem compareceu e quem estava programado para estar presente. Além disso, esses sistemas podem ser usados para automatizar processos de registro, como a emissão de certificados de participação, geração de relatórios de presença, e até mesmo a coleta de feedbacks dos participantes.

\subsubsection{Evolução e Aplicação}
Sistemas de registro de presença não são novos, mas evoluíram significativamente com o tempo. No passado, eram manuais, baseados em listas de papel onde os participantes assinavam sua presença. A digitalização trouxe novas formas de registrar presença \cite{qureshi2024}, como:
\begin{itemize}
    \item Dispositivos biométricos.
    \item RFID (radiofrequência).
    \item Códigos QR.
    \item Aplicativos e plataformas web.
\end{itemize}

Sua aplicação pode ser diversa, como:
\begin{itemize}
    \item \textbf{Eventos corporativos:} controle de presença em treinamentos, workshops, e reuniões.
    \item \textbf{Escolas e universidades:} controle de presença em aulas, seminários, e eventos acadêmicos.
    \item \textbf{Conferências e feiras:} controle de presença em palestras e estandes.
    \item \textbf{Reuniões online:} controle de presença em videoconferências, webinars, e reuniões virtuais.
\end{itemize}

\subsubsection{Funcionalidades e Tecnologias}
Sistemas de presença modernos podem oferecer uma série de funcionalidades, dependendo da complexidade e do público-alvo:

\begin{itemize}
    \item \textbf{Registro pré-evento:} sistemas que permitem inscrições, garantindo que o organizador tenha uma lista prévia dos participantes.
    \item \textbf{Check-in/out:} os participantes confirmam sua presença via QR code, aplicativos, dispositivos RFID, reconhecimento facial ou outros métodos.
    \item \textbf{Relatórios e análise de dados:} esses sistemas permitem gerar relatórios detalhados com dados como horário de chegada, duração da participação, e número de ausentes, oferecendo insights para os organizadores.
    \item \textbf{Integração com outras plataformas:} alguns sistemas se integram com sistemas de gestão de eventos, CRM, e até plataformas de pagamento.
\end{itemize}
