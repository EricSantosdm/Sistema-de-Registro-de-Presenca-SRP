\section{Fundamentação teórica}
\label{sec:fundamentacao_teorica}

\subsection{Introdução aos Sistemas de Registro de Presença}
Sistemas de registro de presença são amplamente utilizados em diversos contextos, como eventos corporativos, conferências, workshops, escolas, e até em reuniões online \cite{ferreira2023}. Eles permitem o controle da participação, oferecendo dados valiosos sobre quem compareceu e quem estava programado para estar presente. Além disso, esses sistemas podem ser usados para automatizar processos de registro, como a emissão de certificados de participação, geração de relatórios de presença, e até mesmo a coleta de feedbacks dos participantes.

\subsubsection{Evolução e Aplicação}
Sistemas de registro de presença não são novos, mas evoluíram significativamente com o tempo. No passado, eram manuais, baseados em listas de papel onde os participantes assinavam sua presença. A digitalização trouxe novas formas de registrar presença \cite{qureshi2024}, como:
\begin{itemize}
    \item Dispositivos biométricos.
    \item RFID (radiofrequência).
    \item Códigos QR.
    \item Aplicativos e plataformas web.
\end{itemize}

Sua aplicação pode ser diversa, como:
\begin{itemize}
    \item \textbf{Eventos corporativos:} controle de presença em treinamentos, workshops, e reuniões.
    \item \textbf{Escolas e universidades:} controle de presença em aulas, seminários, e eventos acadêmicos.
    \item \textbf{Conferências e feiras:} controle de presença em palestras e estandes.
    \item \textbf{Reuniões online:} controle de presença em videoconferências, webinars, e reuniões virtuais.
\end{itemize}

\subsubsection{Funcionalidades e Tecnologias}
Sistemas de presença modernos podem oferecer uma série de funcionalidades, dependendo da complexidade e do público-alvo:

\begin{itemize}
    \item \textbf{Registro pré-evento:} sistemas que permitem inscrições, garantindo que o organizador tenha uma lista prévia dos participantes.
    \item \textbf{Check-in/out:} os participantes confirmam sua presença via QR code, aplicativos, dispositivos RFID, reconhecimento facial ou outros métodos.
    \item \textbf{Relatórios e análise de dados:} esses sistemas permitem gerar relatórios detalhados com dados como horário de chegada, duração da participação, e número de ausentes, oferecendo insights para os organizadores.
    \item \textbf{Integração com outras plataformas:} alguns sistemas se integram com sistemas de gestão de eventos, CRM, e até plataformas de pagamento.
\end{itemize}

\subsection{Fundamentos de Sistema Web}

\subsubsection{Sistemas web}
A Web foi criada em 1989 por Tim Berners-Lee, com o propósito de estabelecer uma plataforma para a troca de informações utilizando a infraestrutura da Internet, baseada no conceito de hipertexto. Desde então, a Web se popularizou rapidamente e se integrou de forma significativa ao cotidiano das pessoas, desempenhando um papel fundamental em diversas atividades diárias \cite{vechiato2019}.

Atualmente, os sistemas web são definidos como aplicações desenvolvidas para a Internet ou redes privadas (intranets), acessíveis por meio de navegadores \cite{ferreira2023}. De acordo com o autor, essas aplicações operam em um ambiente distribuído, onde diferentes partes do programa podem estar localizadas em máquinas distintas. Tradicionalmente, os sistemas web são compostos por conjuntos de páginas HTML e complementados por códigos em linguagens como PHP ou Java.

Ferreira \cite{ferreira2023} destaca que, ao desenvolver sistemas web, é crucial considerar um conjunto de atributos essenciais:

\begin{itemize}
    \item \textbf{Usabilidade:} o sistema deve oferecer uma experiência acessível, com respostas rápidas e informações de alta qualidade.
    \item \textbf{Funcionalidade:} abrange recursos como busca eficiente, recuperação de informações, links, funções e navegação intuitiva.
    \item \textbf{Confiabilidade:} é vital para garantir o funcionamento correto do sistema, com processamento adequado de links, validação de usuários e gestão de possíveis falhas.
    \item \textbf{Manutenibilidade:} o sistema deve ser projetado de forma a facilitar sua manutenção e atualizações ao longo do tempo.
\end{itemize}

Esses atributos são fundamentais para a qualidade e eficácia das aplicações web, assegurando que atendam às necessidades dos usuários e mantenham um desempenho adequado.

\subsubsection{Arquitetura Cliente/Servidor}
A arquitetura cliente/servidor, desenvolvida na década de 1980, é um modelo de computação distribuída que divide as funções do sistema entre clientes e servidores. Nesse modelo, o cliente solicita serviços ou recursos, enquanto o servidor atende essas solicitações, fornecendo os recursos requisitados atravéshfill de uma rede, geralmente a Internet \cite{ferreira2023}.

O cliente atua como a interface do usuário, enviando solicitações para acessar dados ou executar operações, enquanto o servidor processa essas solicitações e retorna os resultados. Exemplos práticos incluem bancos de dados cliente/servidor, onde consultas SQL são enviadas para servidores como MySQL, e servidores de aplicação, onde navegadores web solicitam recursos que são processados por servidores como Apache ou Nginx, entregando conteúdo dinâmico \cite{ferreira2023}.

\subsection{QR Codes}
Atualmente, o QR Code (Quick Response - Resposta Rápida) está se destacando como um novo paradigma que revoluciona o acesso à informação, representando uma evolução significativa do tradicional código de barras \cite{okada2011}. O QR Code é capaz de armazenar dados tanto na vertical quanto na horizontal, permitindo sua leitura a partir de qualquer direção. Após a decodificação, ele pode exibir um trecho de texto ou um link, redirecionando o usuário para um conteúdo online.

A criação de QR Codes é simplificada por uma variedade de ferramentas geradoras disponíveis gratuitamente na internet. Para ler um QR Code, é necessário instalar um aplicativo específico em um smartphone ou tablet. Com o aplicativo instalado, basta apontar a câmera do dispositivo para o código, e o conteúdo, como links, fotos ou mapas, será acessado quase instantaneamente \cite{bernardo2011}.

De acordo com Gonçalves e Cunha \cite{goncalves2014}, uma das principais vantagens do QR Code é a praticidade, pois ele elimina a necessidade de digitar endereços da web, uma tarefa que pode ser complicada em dispositivos móveis. O uso do QR Code simplifica o processo: basta abrir o aplicativo de leitura, apontar para o código, e o conteúdo desejado é exibido automaticamente no navegador.

\subsection{Automatização de Processos}
Automatizar processos envolve racionalizar e otimizar as atividades que geram os resultados de uma organização, visando principalmente simplificar a produção: reduzir o esforço e o tempo necessários para a execução das tarefas, diminuir custos e substituir atividades manuais por aplicações de software \cite{roig2017}. Conforme destaca Roig \cite{roig2017}, “a automação alia tecnologia da informação e o gerenciamento de negócios para otimizar resultados e contribuir para o alcance de objetivos globais”. Nesse contexto, a implementação da automação resulta em uma produção mais ágil, o que aumenta a satisfação do cliente ao receber mercadorias ou serviços em menos tempo.

Além disso, Roig \cite{roig2017} e Martynowicz \cite{martynowicz2018} apontam que a automação de processos proporciona uma série de benefícios, incluindo fácil acesso e circulação de informações, otimização do tempo, e gestão mais segura e eficaz. A automação também contribui para a redução de custos de produção, melhora o monitoramento de resultados em tempo real, e aumenta a produtividade ao permitir a rápida identificação de problemas. Com sistemas integrados, a comunicação se torna mais ágil, a competitividade da organização cresce, e a qualidade de vida dos colaboradores é aprimorada devido à redução do retrabalho. Adicionalmente, a automação promove a integração organizacional e facilita o controle de prazos por meio da rastreabilidade dos processos.

\subsection{Introdução ao Desenvolvimento com Python}
Python se destaca entre as linguagens dinâmicas como uma das mais populares e poderosas, sustentada por uma ampla comunidade global de usuários \cite{borges2010}. Caracteriza-se pela facilidade no entendimento de seus conceitos fundamentais, sendo considerada simples, mas com uma base teórica e técnica complexa. Sua sintaxe clara e concisa promove uma alta legibilidade do código-fonte, o que a torna uma linguagem extremamente produtiva.

Desenvolvida em 1990 por Guido van Rossum no Instituto Nacional de Pesquisa para Matemática e Ciência da Computação da Holanda, Python foi inicialmente voltada para usuários como físicos e engenheiros. O nome ``Python'' foi escolhido em homenagem ao grupo de comédia favorito de van Rossum, Monty Python's Flying Circus, um programa de TV britânico que refletia a abordagem criativa e inovadora que ele queria para a linguagem \cite{meireles2010}.

Segundo Lutz e Ascher \cite{lutz2007}, Python é uma linguagem orientada a objetos amplamente utilizada em uma variedade de domínios, desde programas independentes até aplicações de script. Projetada para otimizar a velocidade de desenvolvimento, Python possui uma sintaxe simples, tipagem dinâmica e ausência de etapas de compilação, além de um conjunto robusto de ferramentas integradas. Estas características permitem aos programadores desenvolverem soluções em uma fração do tempo exigido por outras linguagens tradicionais, resultando em um aumento significativo na produtividade dos desenvolvedores \cite{lutz2007}. Assim, Python combina simplicidade e poder, oferecendo uma base sólida para o desenvolvimento eficiente em diversos contextos. Sua evolução contínua e a adesão de uma comunidade engajada fazem dela uma escolha natural para desenvolvedores que buscam tanto agilidade quanto profundidade técnica.

\subsection{Frameworks de Desenvolvimento Web: Django}
Django é um framework de desenvolvimento web em Python que oferece uma infraestrutura eficiente para criar aplicações com código limpo e sustentável, eliminando a necessidade de reinventar funcionalidades comuns \cite{meireles2010}. Baseado no padrão de arquitetura Model-View-Controller (MVC), Django é um software livre que facilita o desenvolvimento rápido de sites dinâmicos e serviços web.

De acordo com Meireles \cite{meireles2010}, Django possui um sistema de templates que separa a lógica de programação do design, permitindo uma construção ágil de sites e aliviando os programadores de tarefas repetitivas, possibilitando foco em aspectos mais inovadores do desenvolvimento. O framework oferece abstrações de alto nível para padrões comuns, tornando o processo de desenvolvimento mais produtivo e eficiente. Embora seja acessível, seu uso requer conhecimentos básicos de desenvolvimento web, programação orientada a objetos, bancos de dados e Python.

Criado inicialmente para gerenciar um site jornalístico, Django foi lançado como um projeto de código aberto em 2005, com seu nome inspirado no músico de jazz Django Reinhardt, simbolizando a agilidade e inovação do framework \cite{meireles2010}. Django é amplamente utilizado por sua capacidade de acelerar o desenvolvimento e pela clareza que traz na resolução de problemas comuns no desenvolvimento web.

