\section{Abordagem}
\label{sec:abordagem}

Este trabalho apresenta o desenvolvimento de um sistema automatizado para registro de presença via QR code, visando otimizar e simplificar o processo de controle de presença em ambientes acadêmicos. A pesquisa bibliográfica de caráter exploratório e dissertativo foi realizada inicialmente, buscando compreender o contexto acadêmico relacionado ao registro de presença, além de identificar tendências, desafios e soluções que fundamentaram o desenvolvimento da solução.

Na fase de desenvolvimento, foram levantados os requisitos funcionais e técnicos, com a escolha das tecnologias mais adequadas para a construção do sistema. O backend foi implementado utilizando Python e o framework Django, enquanto o frontend utilizou HTML, CSS, JavaScript e o framework Tailwind, garantindo uma interface intuitiva e responsiva. O controle de versão do código foi gerenciado por meio de Git e GitHub.

Para a infraestrutura e deploy, foi utilizada a plataforma \textit{AWS (Amazon Web Services)}, garantindo escalabilidade e segurança. A qualidade do código foi assegurada mediante de testes automatizados com Pytest e ferramentas de análise de qualidade como SonarCloud e SonarQube, que monitoraram possíveis vulnerabilidades e melhorias no código.
