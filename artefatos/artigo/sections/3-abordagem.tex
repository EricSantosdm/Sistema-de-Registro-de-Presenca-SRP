\section{Abordagem}
\label{sec:abordagem}

Este trabalho aborda o desenvolvimento de um sistema automatizado de registro de presença via \textit{QR code}, com o objetivo de otimizar o controle de presença em ambientes acadêmicos. A pesquisa inicial foi bibliográfica e exploratória, focando no contexto do registro de presença, identificando tendências, desafios e soluções relevantes.

Durante o desenvolvimento, foram definidos os requisitos funcionais e técnicos, escolhendo as tecnologias apropriadas. O \textit{backend} foi construído em \textit{Python} com o \textit{framework} \textit{Django}, enquanto o \textit{frontend} utilizou \textit{HTML}, \textit{CSS}, \textit{JavaScript} e o \textit{framework} \textit{Tailwind}, assegurando uma interface intuitiva e responsiva. O gerenciamento de versão do código foi feito com Git e GitHub.

Para a infraestrutura e o deploy, foi empregada a plataforma \textit{AWS (Amazon Web Services)}, que oferece escalabilidade e segurança. A qualidade do código foi garantida por testes automatizados com \textit{Pytest} e ferramentas de análise como \textit{SonarCloud} e \textit{SonarQube}, que monitoraram vulnerabilidades e propuseram melhorias no código.
