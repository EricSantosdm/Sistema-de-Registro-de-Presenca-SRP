\section{Introdução}
\label{sec:introducao}

Com o advento das grandes revoluções industriais, a estrutura social e o estilo de vida da população, especialmente da classe trabalhadora, sofreram mudanças drásticas \cite{hobsbawn2012}. Neste contexto, realizar tarefas de forma mais simples e eficiente tornou-se crucial para a economia de um recurso cada vez mais escasso: o tempo \cite{romera2023}. Atualmente, vivemos na era da informação, caracterizada pelo amplo acesso ao conhecimento, o que impulsionou avanços tecnológicos em diversos setores \cite{silva2019}. Essa evolução exige adaptação e melhoria constante dos processos organizacionais, favorecendo aquelas corporações que adotam práticas de melhoria contínua, garantindo agilidade e flexibilidade no processo produtivo \cite{goncalves2000}.

A automatização, definida como a aplicação de técnicas e conceitos para criar ferramentas que aumentem a eficiência e produtividade a partir de informações recebidas, é uma das principais formas de responder a essas exigências \cite{moraes2012}. No cenário atual, marcado pela crescente demanda por soluções tecnológicas que aumentem a eficiência e a segurança, a automatização de processos como o registro de presença em ambientes corporativos, acadêmicos ou sociais, torna-se indispensável. Métodos tradicionais de registro frequentemente resultam em desperdício de tempo, destacando a necessidade de sistemas mais eficazes \cite{romera2023}.

Nesse contexto, surge a necessidade de um Sistema de Registro de Presença que seja eficiente, flexível e fácil de usar, utilizando tecnologias modernas, como \textit{QR codes}, para automatizar e simplificar o processo de registro. Esse sistema não só garante precisão nos dados coletados, mas também melhora a experiência do usuário, representando um avanço significativo para os organizadores de eventos ao permitir o monitoramento em tempo real, a geração de relatórios detalhados e a tomada de decisões baseadas em dados concretos. O objetivo deste trabalho é desenvolver um Sistema de Registro de Presença via \textit{QR code} no qual se apresenta como uma solução inovadora, que visa otimizar o gerenciamento de presenças de forma centralizada e reduzir significativamente as horas laborais gastas em processos que podem ser automatizados, destacando como essa tecnologia pode transformar o registro de presença em uma atividade mais eficiente e estratégica.

