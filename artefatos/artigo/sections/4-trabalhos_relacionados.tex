\section{Trabalhos Relacionados}
\label{sec:trabalhos_relacionados}

Aranega e Pera \cite{aranega2023} desenvolveram o aplicativo Univap Presença Digital, que utiliza QR codes exclusivos para otimizar a autenticação e a contagem de presença de alunos em escolas. Integrado a uma catraca inteligente e um backend em Node.js, o sistema melhorou significativamente a eficiência e a segurança no controle de acesso, eliminando atrasos e permitindo uma análise detalhada das frequências escolares. A implementação trouxe avanços notáveis na gestão do tempo e na experiência educacional, além de promover práticas mais sustentáveis no ambiente escolar.

Romera e Honorato \cite{romera2023} desenvolveram um aplicativo de registro de presença digital utilizando reconhecimento facial e QR codes. Implementado com tecnologias como Python, React e PostgreSQL, o sistema visa automatizar o controle de presença em ambientes corporativos e acadêmicos, melhorando a eficiência e segurança no processo. Apesar de resultados promissores, o sistema apresentou desafios relacionados à iluminação e à qualidade das imagens no reconhecimento facial, além de questões de segurança no uso dos QR codes.

Paula e Oliveira \cite{paula2023} desenvolveram uma plataforma web para a gestão de eventos acadêmicos, utilizando Laravel, Vue.js e PHP. A plataforma permite o cadastro de eventos e participantes, autenticação de presença via QR code, e a geração de certificados. A solução foi projetada para facilitar a organização dos eventos, reduzir erros e melhorar a eficiência do processo, proporcionando uma experiência mais prática tanto para organizadores quanto para participantes.

